% ----------------------------------------------------------
% Revisão Bibliográfica
% ----------------------------------------------------------
\chapter{Revisão Bibliográfica}
\label{chap:rev}

Este capítulo apresenta um histórico, não necessariamente em ordem
cronológica, dos principais trabalhos envolvendo
o uso de CFD em aplicações nucleares. Além disso, são também apresentados
os trabalhos mais relevantes envolvendo acoplamento neutrônico e
termo-hidráulico utilizando diferentes técnicas, tanto para o cálculo neutrônico
quanto termo-hidráulico.

% Introdução anterior. Isso não cabe aqui, não é revisão bibliográfica.
%
% -----------------------------------------------------------------------------------------------------------------

%Este processo, desde sua metodologia até fundamentos de sua execução, será
%oportunamente - e detalhadamente - apresentado, já que trata-se da principal
%contribuição desta tese.

%O acoplamento entre neutrônica e termo-hidráulica já ocorre de diferentes formas. Na simulação 
%termo-hidráulica são usados códigos de sistemas e códigos de sub-canal. Os códigos de sistemas 
%funcionam modelando os sistemas do reator unidimensionalmente e aplicando as equações básicas 
%para continuidade, momento e energia. O resultado obtido simula o
%comportamento médio dos componentes do reator.
%Estes códigos são normalmente usados em análises de transientes e segurança de reatores. 
%Já os códigos de sub-canal são mais detalhados e, além
%de modelar múltiplos componentes do sistema 
%do reator, são capazes de simular em geometrias tridimensionais \cite{Faghihi2011}. Os códigos 
%CFD substituem o domínio contínuo por um domínio discreto e finito. Neste contexto, as equações 
%que governam o escoamento são integradas sobre todos os elementos que formam o domínio (finito). 
%As integrais obtidas são então discretizadas na forma de um sistema de equações algébricas 
%e, por fim, este sistema é resolvido por métodos interativos \cite{Versteeg2007}.

%No contexto da neutrônica, é possível classificar os códigos em três tipos:
%1) Códigos de difusão ou de transporte, 
%2) Códigos de ordenadas-discretas e 3) Códigos Monte Carlo. Os dois primeiros são determinísticos 
%e o terceiro estocástico. 

%Os códigos de difusão resolvem a equação de difusão de nêutrons. A equação de difusão de nêutrons nada mais é do
%que uma simplificação na modelagem do comportamento dos nêutrons. Uma delas, por exemplo, é a consideração de um
%coeficiente de difusão único representando as direções possíveis dos nêutrons. Em suma, a equação de difusão
%em estado estacionário é obtida de uma relação entre a corrente de nêutrons e o gradiente do fluxo neutrônico,
%representando o fato de que os nêutrons têm uma tendência a migrar de regiões onde são mais numerosos para
%regiões onde são menos numerosos \cite{Hebert2009}. Um dos códigos de difusão 
%mais usados é o PARCS (\textit{Purdue Advanced Reactor Core Simulator}), estando inclusive já acoplado 
%com códigos de termo-hidráulica de sistemas \cite{Xu2006,Barber98}.
%O código usado neste trabalho para o acoplamento é o código milonga
%(grafa-se sem letra maiúscula). O milonga \cite{Theler2014b}
%utiliza o método de volumes finitos na discretização do domínio e disponibiliza a solução
%do cálculo neutrônico pela equação de difusão ou pelo método de ordenadas discretas. Suas características e seu
%funcionamento serão apresentados oportunamente.

%Os códigos de ordenadas discretas resolvem 
%a equação de transporte de \textit{Boltzmann} para o comportamento médio das partículas para então calcular o 
%fluxo de nêutrons. Nesses códigos, o espaço é divido em muitas pequenas caixas e as partículas 
%são movidas entre as caixas. Para geometrias complexas com variações de parâmetros, a preparação de 
%seções de choque exige grande esforço.

%Os códigos de Monte Carlo funcionam simulando as partículas 
%individualmente e gravando aspectos do seu comportamento médio. Devido ao alto custo computacional dos cálculos
%pelo método de Monte Carlo, tardaram a ser usados em cálculos acoplados em comparação a métodos determinísticos.
%Entretanto, nos últimos anos seu uso em sistemas acoplados aumentou consideravelmente \cite{Herman2015, Richard2015, Bennett2016},
%inclusive com o uso de CFD \cite{Leppanen2012}.

%Uma vez apresentados os tipos de códigos mais utilizados para simulações termo-hidráulicas
%e neutrônicas, o próximo passo é entender o porquê de acoplar estes códigos.
% -----------------------------------------------------------------------------------------------------------------
%O acoplamento CFD-neutrônica. Motivação. Vantagens e desvantagens. 
%Nesse capítulo vão todas as referências importantes com explicações 
%de cada uma. 
%\section*{OpenFOAM}
%Por que usá-lo? Vantagens e desvantagens.

%O ponto de partida deste capítulo deveria ser a descrição do problema de acoplamento,
%a contextualização de sua aplicação e as razões que tornam tal problema fundamental
%na simulação de um reator nuclear ou de suas partes. Entretanto, no contexto dessa tese, é impossível
%iniciar as discussões técnicas sem uma exposição, ainda que breve, da expressão que
%aparece no título deste trabalho: \textit{software} livre.




% Revisão propriamente dita.


O crescimento na capacidade de processamento
computacional 
fez com que os pesados cálculos de \textit{CFD} passassem a ser atraentes 
para a Engenharia Nuclear. Mais do que isso, o uso desses códigos passou a ser razão 
de preocupação no sentido de garantir a validade dos seus resultados. O relatório 
da Comissão Regulatória Nuclear (NRC) dos Estados Unidos de 2010 \cite[p.69]{NUREG2010}, 
ao sugerir as melhores práticas e métodos para a atividade de regulação, é bastante claro 
ao apontar a necessidade de tirar proveito da capacidade oferecida por códigos de \textit{CFD}, 
fornecendo inclusive sugestões de parceria com a indústria e universidades objetivando 
desenvolver simulações multidimensionais devidamente acompanhadas de validação e
verificação. O conteúdo desse relatório é, por si só, prova de que a utilização de 
códigos \textit{CFD} é, em definitivo, parte do processo de 
pesquisa e desenvolvimento de reatores nucleares. 

Um exemplo desse uso é a utilização de \textit{CFD} - nesse caso um código proprietário: 
\textit{ANSYS-CFX} - para modelar o fenômeno de ebulição sub-resfriada para o 
desenvolvimento de combustíveis nucleares \cite{Krepper2007}. Devido à capacidade 
dos códigos \textit{CFD} de simular detalhes do escoamento de acordo com a granularidade com que se
divide o domínio de simulação, nesse trabalho esta técnica é utilizada para avaliar 
o fenômeno de fluxo de calor crítico em um elemento combustível. As informações 
fornecidas pelo código \textit{CFD} em relação a fenômenos como rotação, \textit{cross flow} entre 
regiões adjacentes e concentração de bolhas (no caso em que se simule duas fases)
permitem identificar \textit{hot spots}, ou seja, regiões de especial interesse.
Essas simulações permitem avaliar o comportamento
de diferentes projetos de grades espaçadoras de elementos combustíveis \cite{Navarro2011}, por exemplo.

Um trabalho em conjunto entre pesquisadores do ISYRIM, UFMG e CDTN foi conduzido
com o objetivo de investigar o comportamento do reator TRIGA IPR-R1 por um código do tipo \textit{CFD} \cite{Martinez2012}. 
Nesse caso não houve tentativa de acoplamento, mas apenas a simulação simplificada 
do reator utilizando-se o \textit{ANSYS-CFX}. Um fluxo de calor 
foi fornecido para as paredes do combustível obedecendo à uma distribuição axial 
característica do combustível. Na análise quantitativa dos resultados, os autores 
informam que os resultados da simulação numérica apresentam boa concordância com 
dados experimentais coletados durante a operação do reator.

%O acoplamento propriamente dito é feito muitas vezes com um código neutrônico não-determinístico. 
%Esses códigos são baseados no método de Monte Carlo(\cite{mc}) e simulam o comportamento 
%do núcleo do reator através da simulação das histórias dos nêutrons. Isso consiste, basicamente, 
%em cálculos de probabilidades de absorção, choque, escape, etc. de um conjunto de nêutrons. 
%Um exemplo de trabalho de simulação do reator TRIGA IPR-R1 pelo método de Monte Carlo está em \cite{Silva2011}.
%Nesse caso, apenas a neutrônica é simulada.
O uso das técnicas de \textit{CFD} não ficou restrito aos \textit{softwares} ditos proprietários.
Hrvoje Jasak discorreu sobre as razões pelas quais um único \textit{software} monolítico
não era adequado para a solução dos vários problemas modeláveis em \textit{CFD}, tais como requisitos
de usuários específicos para um certo problema que poderiam envolver propriedades
experimentais, equações adicionais ao problema, ou acoplamento com pacotes externos.
O autor ressalta, ainda, que para diferentes problemas, diferentes formas de discretização
e soluções numéricas são necessárias. Além das suas críticas às dificuldades de uso e
não-flexibilidade dos códigos fechados, propôs uma solução: uma biblioteca
orientada a objetos para simulações numéricas escrita em linguagem C++ e, mais do que tudo,
licenciada como \textit{software} livre \cite{Jasak2007} chamada \textit{OpenFOAM}.
Nesse trabalho o autor também desfaz as polêmicas sobre o desempenho de códigos escritos
em C++ e demonstra como as equações são modeladas dentro do \textit{OpenFOAM}. Um
exemplo é a equação da energia cinética turbulenta no modelo RANS
\begin{equation}
  \frac{\delta k}{\delta t} + \nabla . (\mathbf{u}k) - \nabla . [(\nu + \nu_{t})
    \nabla k] = \nu_t \left[ \frac{1}{2}(\nabla \mathbf{u} + \nabla \mathbf{u}^T)\right]^2 - \frac{\epsilon_0}{k_0}k
\end{equation}
que é representada como código no \textit{OpenFOAM} do seguinte modo:
\begin{lstlisting}[language=c++]
solve
(
   fvm::ddt(k)
 + fvm::div(phi, k)
 - fvm::laplacian(nu() + nut, k)
== nut*magSqr(symm(fvc::grad(U)))
 - fvm::Sp(epsilon/k, k)
);  
\end{lstlisting}

Estes exemplos ajudam a ilustrar a relativa simplicidade - de acordo com o autor, mesmo
alguém com conhecimento limitado em programação e sem referência à programação orientada
a objetos em C++ é capaz de entender por alto o trecho apresentado - da modelagem
de equações diferenciais no \textit{OpenFOAM}.

Vale ressaltar que, devido ao custo computacional dos cálculos \textit{CFD}, seu uso na indústria
nuclear é, se comparado a outros métodos de simulação termo-hidráulica, bastante recente. Um artigo que
veio de forma definitiva atestar a importância dos cálculos \textit{CFD} na indústria nuclear
foi apresentado por Emilio Baglietto \cite{Baglietto2011}. Nesse artigo, o autor afirma
que o uso de \textit{CFD}, só ou combinado com códigos neutrônicos e ferramentas
de análise, aumentou a capacidade e disponibilidade dos reatores nucleares
atualmente em operação. Ele afirma,
ainda, que tal tecnologia de análise está ajudando no desenvolvimento dos reatores
de quarta geração, especialmente nesse caso, em que dados experimentais e operacionais
estão menos disponíveis. Vendedores e operadores de reatores utilizarão muito mais
modelagem, simulação e computação para determinar como reatores e seus componentes
irão se portar quando em operação. O autor é categórico ao afirmar que ``em um
período de 20 anos, a indústria passará dos métodos convencionais baseados em
correlações experimentais ao uso de \textit{CFD} para pequenos componentes, grandes componentes
e análise de uma planta completa''. 

Entretanto, para obter resultados os mais realistas possível, é necessário simular 
tanto os fenômenos termo-hidráulicos quanto neutrônicos. Isso 
se deve ao fato de que os principais fatores que influenciam na distribuição do fluxo de 
nêutrons no núcleo de um reator nuclear são as propriedades do moderador e refrigerante. No caso de um
reator do tipo PWR, a água exerce ambos os papéis, sendo também usada grafita para 
a moderação. Pequenas variações na temperatura, densidade e composição 
da água podem mudar consideravelmente o fator de multiplicação de nêutrons ($k_{eff}$). Daí a importância e 
a necessidade de acoplar os cálculos termo-hidráulicos - as variações nas propriedades da água - com as variações 
no fluxo neutrônico.

%O já mencionado crescimento na capacidade de cálculo dos computadores pessoais permitiu não só o uso 
%de técnicas mais elaboradas de cálculos termo-hidráulicos bem como seu uso em conjunto 
%com códigos de cálculo neutrônico.

Um dos primeiros trabalhos diretamente focados no 
acoplamento termo-hidráulico foi apresentado por \cite{Barber98}. Esse trabalho 
é essencialmente um relatório técnico, no qual o foco está na construção de uma interface 
genérica de acoplamento do código PARCS com códigos de termo-hidráulica. Apesar de não 
gerar resultados de simulações, esse documento evidencia a importância e o interesse 
no procedimento de acoplamento, além de fornecer preciosas informações acerca da 
implementação de estruturas de dados, manipulação de entrada e saída, troca de mensagens, 
uso de biblioteca de troca de mensagens PVM \cite{Geist94}, dentre outros pontos cruciais
da implementação de software. Cabe ressaltar que esta interface, 
chamada \textit{General Interface} é ainda usada pelo código PARCS nas suas versões mais 
recentes. 

O acoplamento das simulações de diferentes fenômenos físicos tem sido também chamado de
\textbf{multi-física}. 
Uma definição prática para multi-física ou física-acoplada é dada por Paul Lethbridge \cite{Lethbridge2005}, definindo que 
a multi-física é, em essência, a análise de fenômenos físicos distintos de forma combinada.

No texto desta tese, 
por conveniência, a palavra acoplamento será usada tanto em referência ao processo de análise dos fenômenos físicos conjuntamente 
quanto em referência à implementação do software para essa tarefa. Nos casos em que possa haver ambiguidade, 
a definição será explicada de modo a não deixar margens a dúvidas.

Vale ressaltar, já que nos referimos à implementação do acoplamento em software, que uma aplicação multi-física 
pode levar entre 4 e 6 anos para ser efetivamente útil, podendo chegar a uma vida útil de várias décadas 
\cite{Graham2004}. Isto posto, é possível concluir que o esforço em construir o acoplamento é compensado 
pela possibilidade de uso do código durante vários anos.

Ainda sobre a importância recente dos cálculos neutrônicos e termo-hidráulicos acoplados, foi iniciado em 2012
no Centro de Pesquisas Técnicas VTT (Finlândia) o projeto \textit{Numerical Multi-Physics (NUMPS)} \cite{Leppanen2015}.
Dentre seus objetivos, estão o desenvolvimento do código de física de reatores \textit{Serpent}, que utiliza o método de
Monte Carlo e do código PORFLO, do tipo \textit{CFD}. Esse desenvolvimento oferece oportunidades para a educação das novas
gerações de especialistas, não apenas com compreensão de teoria e métodos, mas também com entendimento ao nível de
código-fonte das ferramentas de cálculo. Com um viés diferente, o consórcio para a simulação avançada de reatores
de água leve (CASL), criado em 2010, tem como objetivo prover capacidade de modelagem e simulação para apoiar e acelerar
a melhora na competitividade econômica e redução do volume de combustível gasto por unidade de energia. Alguns de seus
resultados são sistemas multi-física de alto desempenho com o propósito de atacar problemas antigos da indústria nuclear,
tais como predição de comportamento de desgaste de revestimento de combustíveis e estudo da química da corrosão em várias
escalas \cite{Turinsky2016}.


% Citar os trabalhos do IVANOV aqui...
Talvez o mais importante trabalho especificamente dedicado ao acoplamento 
neutrônico e termo-hidráulico seja o de Kostadin Ivanov \cite{Ivanov2007}. Nele são analisados e classificados diferentes tipos de 
acoplamento, seus componentes e suas aplicações. Os diversos parâmetros do acoplamento são esquematicamente divididos em: 
\begin{enumerate}
\item \textbf{Forma de acoplamento}: externo, quando o código neutrônico é combinado com parte do código termo-hidráulico, 
geralmente na forma de condições de contorno, e então acoplado ao sistema termo-hidráulico completo. O acoplamento 
é definido como interno quando a neutrônica é integrada ao modelo de transferência de calor do sistema termo-hidráulico. 
Em outras palavras, o acoplamento interno é uma implementação multi-física. Deve-se dizer que não há, na literatura, uma
concordância completa sobre a nomenclatura para a forma de acoplamento. 
\item \textbf{Abordagens de acoplamento}: integração em série ou processamento em paralelo. Na integração em série o algoritmo 
para cálculo neutrônico é integrado como uma rotina ao sistema termo-hidráulico e então executado sequencialmente 
após os cálculos termo-hidráulicos. Na abordagem de processamento em paralelo, geralmente a troca de dados é 
intermediada por um sistema de troca de mensagens como PVM \cite{Geist94} ou MPI \cite{Quinn2004}. Nesse caso, os 
módulos de neutrônica e termo-hidráulica têm capacidade separada de execução e resposta, o que permite uma execução 
mais eficiente do ponto de vista computacional.
\item \textbf{Sobreposição espacial de malhas}: pode ser fixo, quando um canal termo-hidráulico representa um canal neutrônico, 
ou flexível, quando são usados ou especificados esquemas de mapeamento. Um esquema avançado de mapeamento 
e interpolação de malhas \cite{Beaudoin2008} é implementado em alguns dos modernos códigos de \textit{CFD}.
O uso de esquemas de mapeamento de malhas trazem implicações, tais como arredondamento e truncamentos no pareamento
entre céluas, além de custo computacional relativo a percorrer as malhas ou armazenamento de estruturas de mapeamento
em memória.
\item \textbf{Algoritmos de controle de \textit{time-step}}: os transientes neutrônicos são geralmente muito mais rápidos que 
os transientes termo-hidráulicos. A utilização de um único \textit{time-step} longo pode levar à não detecção de transientes 
rápidos e no caso oposto ao desperdício de recursos computacionais ao se simular eventos indistinguíveis repetidamente.
\item \textbf{Acoplamento numérico}: o autor se refere nesta classificação ao tempo de troca de informações entre o modelo 
neutrônico e termo-hidráulico. Pode ser explícito, semi-explícito e implícito, dependendo da forma como cada esquema 
realiza o cálculo das variáveis do \textit{time-step} atual baseado em variáveis do \textit{time-step} atual ou 
anterior.
\item \textbf{Esquemas de convergência do acoplamento}: definidos de acordo com a forma com que a simulação é considerada 
finalizada. Se há apenas uma estimativa para a convergência de ambos os códigos, o esquema é considerado fracamento 
acoplado, por exemplo.  
\end{enumerate}

Em relação ao último item, poucos autores se dedicam a uma análise sistemática da forma como se
dá a convergência nos cálculos acoplados. Usualmente são apenas apresentados os esquemas
de convergência e o foco das análises se concentra nos resultados dos cálculos acoplados.
Zerkak \cite{Zerkak2015}, por sua vez, faz uma revisão dos métodos de acoplamento com
foco em como melhorar a convergência das atuais técnicas de acoplamento. O autor faz uma
análise pormenorizada da técnica de \textit{operator splitting} e várias de suas melhorias,
além de analisar vários outros esquemas numéricos de solução de \textit{PDE's}. Sua análise visa
uma avaliação de vantagens e desvantagens dos métodos numéricos em relação à eficiência
computacional, convergência e esforço de implementação e modularidade, sempre com vistas
à análise de reatores nucleares.

O autor ainda comenta sobre a importância da geração de seções de choque adequadas aos transientes esperados na 
simulação acoplada e sua interdependência. Na presente tese foram geradas seções de choque para a obtenção dos
coeficientes para a solução da equação de difusão para dois grupos de nêutrons.
Uma breve explicação de como seções de choque para poucos grupos podem ser geradas pode ser encontrada no
artigo de Friedman \cite{Friedman2013}. Cabe ressaltar que um grande número de trabalhos encontrados
na literatura tem utilizado repetidamente o código \textit{Serpent} \cite{Serpent2013} para
a geração de seções de choque, tanto para cálculos estocásticos quanto determinísticos \cite{Jareteg2014}. O leitor interessado nos detalhes dos métodos utilizados no \textit{Serpent} na
preparação de seções de choque para os cálculos de transporte de nêutrons pode encontrar mais informações no trabalho de Leppänen \cite{Leppanen2009}.

Dorval \cite{Dorval2015} investiga
a geração de seções de choque para poucos grupos a partir do código \textit{Serpent} para
o cálculo neutrônico de reatores rápidos refrigerados a sódio. Além dos coeficientes de
difusão padrão gerados pelo \textit{Serpent}, é proposto um novo método para cálculo de
coeficientes de difusão direcionais. 

Confirmando o atual interesse em cálculos \textit{CFD} no domínio nuclear, um trabalho de simulação envolvendo
neutrônica e termo-hidráulica com uso de \textit{CFD} propõe a simulação de reatores avançados refrigerados
a gás \cite{Hossain2011}. O modelo utilizado 
para a simulação neutrônica foi o de cinética pontual devido à sua simplicidade. No modelo 
de cinética pontual assume-se que a forma do perfil do fluxo de nêutrons durante um transiente 
não varia, mas apenas os valores do fluxo variam com o tempo. A variação no fluxo é dada 
pelo balanço entre nêutrons produzidos e perdidos, considerando seis classes de precursores 
para nêutrons atrasados. O desenvolvimento matemático leva a um sistema de sete equações diferenciais 
ordinárias, conhecidas como equações de cinética pontual. A implementação da solução desse sistema 
é, nesse caso, feito dentro do código termo-hidráulico utilizado (\textit{TH3D}).

Em outro trabalho que se utilizou de \textit{CFD} \cite{Yan2011}, foi simulado o comportamento do escoamento em um feixe de elementos 
combustíveis e espaçadores utilizando-se de \textit{CFD} acoplado à neutrônica. A neutrônica 
simulada é baseada no \textit{method of characteristics (MOC)}, que evita a necessidade de geração 
de constantes para poucos grupos \textit{a priori}. A abordagem usada no 
acoplamento é do tipo externa, com o código neutrônico DeCART e o código termo-hidráulico 
STAR-CCM+ executando sequencialmente e escrevendo e lendo arquivos em disco em um 
diretório comum. São dois critérios para finalizar a simulação: o código DeCART se baseia 
na convergência da fonte de fissão, enquanto o código STAR-CCM+ no resíduo da energia. No que toca à 
discretização espacial (sobreposição espacial de malhas), é usado um mapeamento entres as malhas 
utilizadas nos dois códigos. Apesar de diferenças nas malhas, as fronteiras materiais, ou seja, geometria, 
fronteira e posição de diferentes materiais são equivalentes em ambos os modelos, simplificando
o esquema de mapeamento. Suas conclusões reiteram a importância da técnica de \textit{CFD} na indústria nuclear. Em particular, 
a capacidade de verificar os efeitos da grade misturadora na temperatura e densidade do refrigerante 
levando-se em consideração os efeitos neutrônicos. Nas palavras do autor, esse acoplamento 
``permite um melhor entendimento da margem de DNB (\textit{Departure from Nucleate Boiling})''.
O autor finaliza apontando o interesse em incrementar a simulação 
agregando outros aspectos físicos, como um modelo de corrosão, um modelo de geração do particulado, 
um modelo de interação revestimento/pastilha e outros. Percebe-se aqui que as diversas aplicações 
de multi-física já são uma realidade, e não mais uma aposta ou uma perspectiva futura.

Corroborando com a ideia da importância dos cálculos acoplados, Faghihi faz uma análise
dos diversos tipos de códigos usados para cálculos termo-hidráulicos no que se refere à
forma como os sistemas são modelados (códigos de sistemas, códigos de sub-canais e códigos
do tipo \textit{CFD}) e outra análise semelhante para os códigos neutrônicos \cite{Faghihi2011}. Entretanto, seu
foco está além da descrição de conceitos de acoplamento
neutrônico e termo-hidráulico. O autor propõe o uso de redes neuronais treinadas com dados de
outras simulações para a solução do fluxo neutrônico em reatores do tipo LWR.

Talvez o mais completo trabalho relativo ao acoplamento neutrônico e termo-hidráulico com uso do 
\textit{CFD} \textit{OpenFOAM} seja a tese de Klas Jareteg \cite{Jareteg2012}. Nela, o acoplamento é feito usando o mesmo 
software para a simulação termo-hidráulica e neutrônica. A metodologia utilizada pode ser brevemente 
resumida em alguns passos principais: 1) criação de uma malha adequada, 2) discretização das 
equações descrevendo o problema, 3) definição das condições de contorno, 4) geração dos 
parâmetros neutrônicos (por exemplo, seções de choque macroscópicas) e 5) solução do 
problema neutrônico e termo-hidráulico de forma acoplada. O autor usa um dos modelos de solução presentes no 
software de \textit{CFD} \textit{OpenFOAM} \cite{OpenFOAM2013} e 
altera esse modelo que simula um escoamento turbulento de um fluido compressível com transferência de energia
por radiação térmica (\texttt{buoyantSimpleRadiationSolver}) adicionando uma implementação da equação de difusão multi-grupos 
ao modelo. Isto é feito utilizando a capacidade geral de discretização e solução de equações bem como os algoritmos 
para solução numérica já presentes no \textit{OpenFOAM}. Dentre suas conclusões, estão que o acoplamento iterativo 
utilizado é funcional e estável, a solução para a neutrônica necessitou de menos iterações para convergir 
do que que a solução pressão-velocidade e que a termo-hidráulica precisa de malhas mais refinadas do que a solução 
neutrônica. A qualidade desse trabalho e muitas das soluções adotadas servem como referências às implementações 
e simulações a serem feitas nesta tese.

O referido trabalho de tese deu origem a um artigo \cite{Jareteg2014} no qual o acoplamento
neutrônico e termo-hidráulico é classificado como interno e é utilizada uma malha
refinada. O objetivo desse artigo é provar que
tal acoplamento é factível e gera resultados com um nível de detalhes sem precedentes. Além
disso, os autores apontam que cálculos com malhas menos refinadas geram discrepâncias
de até 0,5\% na potência por vareta e várias dezenas de \textit{pcm} no fator de multiplicação.
Os autores justificam, ainda, que apesar de a neutrônica não necessitar de uma malha
tão detalhada quando necessita a termo-hidráulica, seu uso se justifica pela não necessidade
de interpolações ou mapeamentos entres os elementos da malha. Além disso, como trata-se
de um acoplamento interno, não há necessidade de transferência de dados, poupando-se
tempo de processamento.
Nesse trabalho o problema é tratado como de regime permanente, bem como o problema escopo
desta tese.


O \textit{OpenFOAM} também foi a ferramenta escolhida por pesquisadores do Paul Sherrer
Institut para reduzir os esforços de desenvolvimento ao prover rotinas de discretização
e capacidade de execução em paralelo de soluções de equações diferenciais. A partir
de \textit{solvers} disponíveis no \textit{OpenFOAM} foi implementado um \textit{solver}
multi-física batizado \textit{General Nuclear Foam - GeN-Foam} \cite{Fiorina2015}.
Esse \textit{solver}
possui a capacidade de resolver escoamentos compressíveis ou incompressíveis baseados
no modelo de turbulência $\kappa-\epsilon$, sendo também capaz de resolver
problemas em malhas menos refinadas utilizando o modelo de meios porosos. Tem, ainda,
implementado um \textit{sub-solver} de termomecânica e outro \textit{sub-solver} para
a solução da equação de difusão para multi-grupos. Este último, provê ainda a capacidade
de resolver problemas com malhas móveis e tem implementados métodos de leitura de seções
de choque geradas pelo código Serpent \cite{Serpent2013}. Os autores classificam
o acoplamento implementado como semi-implícito e como são usadas malhas diferentes para
os modelos termo-hidráulico, termomecânicos e de difusão de nêutrons, é feito um mapeamento
consistente entre as três malhas através do uso de um algoritmo de peso, célula e volume
implementado pelo próprio \textit{OpenFOAM} \cite{OpenFOAM2015}. A aplicação inicial desse
\textit{solver} foi na simulação do reator rápido a sódio europeu (ESFR). Mesmo com as
simplificações no modelo, as respostas numéricas para as simulações de dois transientes
testados ficaram próximas das obtidas pelas simulações de referências feitas com o
código TRACE.

Uma das razões do uso do acoplamento, cujo objetivo é obter dados mais precisos e realistas, é na 
análise de segurança de reatores. Em projetos de reatores inovadores já são feitos cálculos de multi-física para 
análise de acidentes. No trabalho de Lázaro \cite{Lazaro2013}, um reator de IV geração refrigerado a sódio 
é simulado de forma acoplada. Códigos já usados na análise de reatores resfriados a água leve são adaptados 
para o uso em reatores de nêutrons rápidos resfriados a sódio. Um ponto-chave, segundo o autor, é ser capaz 
de simular os fenômenos presentes na operação de reatores em três dimensões. Isso se justifica, ainda nas 
palavras do autor, devido a possíveis componentes assimétricos em transientes hipotéticos e que 
a modelagem unidimensional com cinética pontual não é capaz de reproduzir. Assim como em vários trabalhos 
sobre o tema de acoplamento, o código usado para a geração de seções de choque homogeneizadas 
foi o \textit{Serpent} \cite{Serpent2013}. Nesse trabalho, o código \textit{Serpent} foi ainda usado 
para cálculo neutrônico do núcleo e validação do código PARCS \cite{PARCS2006}. Sua conclusão nesse trabalho 
é de que as adaptações feitas nos códigos levaram a resultados consistentes, mas que ainda há trabalho 
a ser feito para a simulação tridimensional completa desse tipo de reator.

O pioneirismo do grupo da universidade de Chalmers, em Gotemburgo - Suécia, encabeçado
por Christophe Demazière, além dos já citados trabalhos \cite{Jareteg2012, Jareteg2014},
também se refletiu em outras formas de acoplamento. Desta vez, sem o uso do \textit{OpenFOAM},
foi implementada uma ferramenta para estimar flutuações no fluxo neutrônico, temperatura
do combustível, densidade do moderador e velocidade do escoamento em um reator do tipo PWR
\cite{Larsson2012}. Este tipo de ferramenta visa captar as flutuações em parâmetros
essenciais do funcionamento do reator e, então, encontrar eventuais anomalias. A validação
foi feita com dados de um reator PWR comercial e simulações feitas com os códigos
RELAP e PARCS acoplados bem com o \textit{CFD} comercial \textit{Fluent}. Estas simulações
foram feitas para um modelo tridimensional sem mapeamento, ou seja, ambos os códigos
utilizaram malhas idênticas, num acoplamento um para um entre neutrônica e termo-hidráulica.

Ainda na mesma linha da utilização do \textit{OpenFOAM} como plataforma única para neutrônica e
termo-hidráulica, Klaus Jareteg estendeu seu trabalho à simulação de elementos combustíveis de
reatores do tipo PWR \cite{Jareteg2015}. Neste trabalho, é proposto um mapeamento entre os valores
calculados de seções de choque - mais uma vez utilizando-se do \textit{software Serpent} - gerados
para uma região geométrica para células da malha. Isso se dá pelo uso de um \textit{script} em
linguagem \textit{Python} que faz o mapeamento uma única vez na inicialização do problema e
a partir daí utiliza os dados armazenados. Este sistema acoplado utiliza-se de forma elegante da
capacidade fornecida pelo próprio \textit{OpenFOAM} para decomposição de malhas e execução em paralelo.
Com isso, o sistema pode ser escalado de acordo com o poder computacional disponível de forma
absolutamente transparente. Pode-se dizer que este sistema é a evolução quase natural do trabalho
apresentado pelo autor na sua tese de doutorado \cite{Jareteg2012}.

%Cabe também ressaltar que os trabalhos
%do grupo da Universidade de Gotemburgo focam no uso de uma malha fina de modo a resolver o sistema
%acoplado com detalhes, assim como o sistema sem mapeamento proposto nesta tese.


A importância dada aos cálculos de neutrônica e termo-hidráulica acoplados
é ratificada no trabalho de Jaakko Leppänen sobre o código de Monte Carlo
\textit{Serpent} \cite{Leppanen2012}. Até pouco tempo, a exigência computacional dos códigos
de Monte Carlo impossibilitavam seu uso em cálculos acoplados e iterativos. O aumento
quase exponencial na capacidade computacional jogou por terra tais restrições.
Com os códigos de Monte Carlo executando mais rapidamente - em especial o código \textit{Serpent-2} -
o autor optou por oferecer uma interface acoplável por padrão neste \textit{software}.
No trabalho em questão, é implementada uma interface multi-física para acoplamento com um
código termo-hidráulico externo. Atualmente, o \textit{Serpent-2} conta com uma interface para utilização
de malhas não-estruturadas no formato \textit{OpenFOAM}.

A variedade de formas de acoplamento parece não parar de crescer. Miriam Vazquez
\cite{Vazquez2012} propõe
um sistema acoplado utilizando também o método de Monte Carlo, neste caso com o
\textit{software} MCNPX, juntamente com o código de sub-canais COBRA-IV. Nessa primeira
abordagem, os autores fazem algumas simplificações, como média no sub-canal para a
densidade do moderador, por exemplo. O intercâmbio de dados é feito na forma de arquivos
texto, um método rudimentar mas funcional. Seções de choque são usadas de três formas
diferentes, permitindo estimar alargamento \textit{Doppler}. Este trabalho tem foco na simulação
de um reator rápido a sódio, o que reforça o argumento do uso de cálculos acoplados no projeto
de reatores inovadores.

Um trabalho que merece a atenção trata do acoplamento de neutrônica e termo-hidráulica com o objetivo
de calculas transientes rápidos num reator conceitual que é uma variação de um reator do tipo CANDU
\cite{Hummel2016}. Apesar de não trazer inovação na forma como se dá o acoplamento - uso de arquivos
texto para leitura e escrita de dados intercambiados com cálculos separados de cada \textit{software} -
este trabalho amplia o uso de sistemas de cálculos em estado estacionário para investigar situações
de transientes rápidos no sistema conceitual.



Os exemplos de tempo investido em sistemas de cálculos acoplados não cessam. O grupo
de métodos multi-física do \textit{Idaho National Laboratory} desenvolveu o sistema
\textit{MOOSE: Multiphysics Object Oriented Simulation Environment}. Este sistema objetiva
a solução de problemas fortemente acoplados ao mesmo tempo em que fornece suporte para
o desenvolvimento de, segundo o autor, ``engenharia de aplicações de análise''
\cite{Gaston2009}. O foco do seu desenvolvimento vai além do desempenho:
\begin{itemize}
\item Desenvolvido objetivando execução em paralelo \textbf{massiva} e escalabilidade,
  além de facilidade de uso das aplicações;
\item Utiliza-se de modernos princípios de desenvolvimento de \textit{software}, de modo
  a permitir manutenção e extensões de forma econômica;
\item As aplicações devem obedecer rigorosas especificações de verificação e validação,
  implicando no mesmo tipo de rigor para o próprio sistema.
\end{itemize}
Estas características tornam o \textit{MOOSE} a plataforma de computação multi-física
de base do \textit{Idaho National Laboratory}, oferecendo suporte ao desenvolvimento
de projetos e códigos de análise paralelos e multidimensionais.

De carona na plataforma \textit{MOOSE}, não tardaram a aparecer aplicações para
este \textit{framework}. Uma aplicação para a solução da equação de transporte
de nêutrons em elementos finitos tridimensional, chamada \textit{RATTLES$_{N}$AKE},
foi implementada. Para a queima de combustível nuclear, nos mesmos moldes e
também utilizando elementos finitos em malha tridimensional, foi criada a aplicação
\textit{BISON}. Esta aplicação resolve implicitamente as equações termomecânicas
acopladas, oferecendo modelos para inchaço e densificação do combustível, dentre outros.
Sendo ambas as aplicações desenvolvidas sobre a plataforma \textit{MOOSE}, o acoplamento
entre ambas foi direto \cite{Gleicher2014}.
Os resultados dos cálculos acoplados para uma vareta combustível foram validados por uma simulação com o código Monte Carlo \textit{Serpent},
os cálculos acoplados feitos com malhas de diferentes granularidades, e apontaram boa
concordância com os resultados do \textit{Serpent}.

Apesar de não ser um sistema \textit{CFD}, mas de elementos finitos, a razão de
apresentar tanto a plataforma \textit{MOOSE} quanto um exemplo de suas aplicações
acopladas é o de atestar que a solução de problemas do domínio nuclear de forma
acoplada não é hoje uma possibilidade, mas efetivamente uma realidade.

Vindo a reforçar esta ideia, o trabalho de Rodney Schmidt \cite{Schmidt2015} apresenta o acoplamento
de forma ainda mais ambiciosa. Neste trabalho, patrocinado pelo Departamento de Energia dos Estados
Unidos (DOE), é proposto um ambiente virtual para aplicações em reatores. O usuário é capaz de definir
seu fluxo de trabalho obedecendo a regras simples e, a partir disto, realizar simulações do início ao
fim usando uma variedade de aplicações distintas que se comunicam fazendo o cálculo multi-física. Algumas
das aplicações são de neutrônica, termo-hidráulica, avaliação de desempenho de combustíveis e química
dos reatores. Este é, até a presente data, um dos trabalhos de acoplamento com um objetivo mais amplo
em termos de variedade de aplicações a serem usadas.

Recentemente foi apresentada uma metodologia de mapeamento orientada a objetos chamada
SMITHERS \cite{Richard2015}. Consiste num sistema acoplado que faz mapeamento \textit{on-the-fly}
da distribuição de potência nos materiais num sistema de transportes de nêutrons baseado
em MCNP para um outro código responsável pelos cálculos termo-hidráulicos. É uma metodologia
complexa, que envolve cálculo de potência, queima, e utiliza um mapeamento robusto entre
a representação nodal da maioria dos programas de cálculo termo-hidráulico e a geometria
do código MCNP. Também faz uso de \textit{scripts} em linguagem \textit{Python} para
tratamento do mapeamento além de estruturas de dados próprias dessa linguagem de programação
para conseguir inserções e remoções em memórias a custo $O(1)$
\footnote{Sobre a notação $O(n)$, amplamente utilizada em Ciência da Computação, uma
  referência é o livro Concrete Mathematics \cite[Seção~9.2]{Graham1994}}.
% Talvez eu devesse citar o livro do Cormen nessa citação big O
Os primeiros testes do uso do SMITHERS foram feitos para reatores do tipo BWR e
PWR.

Alguns dos trabalhos apresentados na literatura, apresentam o que se poderia dizer,
não sem certa pretensão, uma segunda ``geração'' de sistemas acoplados. Em especial os trabalhos
envolvendo a plataforma \textit{MOOSE} (e as aplicações feitas para serem usadas em seu
\textit{framework}) e, em uma escala um pouco menor mais ainda acima de acoplamentos específicos,
a metodologia SMITHERS. Entretanto, há especificamente três trabalhos que se destacam por razões específicas que
merecem menções não apenas pela forma como estão relacionados com o trabalho desenvolvido nesta tese, mas pela
potencial em caminhos de desenvolvimento de sistemas acoplados.

Desta lista mencionada, o primeiro trabalho consiste em um sistema neutrônico e termo-hidráulico
acoplado \cite{Bennett2016}. Neste acoplamento, são usados um código
de Monte Carlo (\textit{MCNP6}) para os cálculos neutrônicos e um código de sub-canais (\textit{CTF}). Apesar de não apresentarem
inovações grandes em relações a acoplamentos anteriores, há duas coisas para as quais deve ser chamada a atenção.
O \textit{MCNP6} é capaz de gerar seções de choque \textit{on-the-fly}, uma característica desenvolvida com objetivo
de oferecer suporte ao acoplamento termo-hidráulico. Há alguns anos, o uso de códigos de Monte Carlo para
acoplamento era impensável. Entretanto, nove anos depois do principal trabalho referência em acoplamento \cite{Ivanov2007}, não só temos
diversos exemplos do uso de códigos de Monte Carlo acoplados, como dois dos principais deles (\textit{MCNP} e \textit{Serpent})
já possuem por definição de projeto, formas de serem acoplados. A segunda razão que vale mencionar neste
trabalho, pode passar despercebida: como já mencionado, são nove anos entre dois trabalhos no mesmo tema.
Isso serve como exemplo do que é hoje a execução de cálculos acoplados e a insistência em
investigar formas de fazê-la cada vez mais robusta. A multi-física é hoje uma realidade.

Outro trabalho que merece uma atenção especial é o de Bryan Herman \cite{Herman2015}. Neste caso, novamente
é usada um código de Monte Carlo para o acoplamento com a termo-hidráulica. A razão de apresentar este trabalho
não reside na tentativa de demonstrar a importância dos cálculos multi física - espera-se que, neste ponto,
o leitor já esteja convencido. O trabalho consiste principalmente em formas da aceleração da
convergência das fontes de fissão, de modo a acelerar a obtenção do resultado final enquanto
diminui-se o gasto de memória. Apesar da inovação na forma de acelerar os cálculos, a razão para uma menção
especial deste trabalho não está no conteúdo técnico, mas no \textit{software} de Monte Carlo em si.
Neste trabalho é usado o código \textit{OpenMC} \cite{Romano2013}. Desenvolvido no \textit{Massachusetts
  Institute of Technology} (MIT), é um código aberto oferecido com uma licença específica do próprio MIT.
Esta licença garante a qualquer pessoa obter o código-fonte desse \textit{software} e sua
documentação associada sem custo. É também garantido o direito de usá-lo sem restrição.
E é este o ponto em que se deseja dar ênfase: diferentemente da maioria
dos \textit{softwares} usados e disponíveis na indústria nuclear (leia-se indústria também englobando a academia)
este é completamente aberto e livre. Um sinal de que a mudança de paradigmas na
forma como se produz e utiliza \textit{software} também chegou à área nuclear.

Feita a apresentação anterior, o terceiro trabalho que merece ênfase, já não pode ser considerado
especificamente um sistema acoplado. O sistema \textit{Cyclus} \cite{Huff2016} é certamente muito mais
do que isso. Este sistema é um \textit{framework} completo de simulação do ciclo do combustível nuclear.
Seu desenvolvimento é, desde o projeto inicial, feito em forma de \textit{software} livre. Seu objetivo é ser um sistema aberto,
flexível, robusto e de uso geral, ao contrário das soluções existentes baseadas em sistemas fechados
definidos para conjuntos de reatores específicos.
No \textit{paper} citado, os autores explicam os conceitos do projeto e desenvolvimento do \textit{Cyclus}, com
foco no uso de ferramentas abertas e desenvolvimento colaborativo e comunitário. Nesse sentido, um sistema
como o \textit{Cyclus} é, no mínimo, uma inspiração para o desenvolvimento de outros sistemas ou códigos
com o mesmo objetivo: de desenvolvimento aberto, comunitário e colaborativo. É este um dos objetivos
desta tese: trazer as técnicas de desenvolvimento aberto, mas também oferecer o resultado obtido -
um sistema acoplado de cálculos de reatores - a todos e quaisquer interessados em aprender, melhorar
e contribuir com seu desenvolvimento.

Concluindo, nesse capítulo tentou-se dar uma visão geral do que é \textit{software} livre e do
seu uso nas diversas áreas do conhecimento humano, inclusive em situações de aplicações
críticas e na indústria nuclear. Chamou-se atenção para a importância que vêm ganhando as
técnicas \textit{CFD} na indústria nuclear e, sobretudo, seu recente uso na termo-hidráulica acoplada aos cálculos
neutrônicos. A lista de aplicações citadas é certamente não-exaustiva e há razões para
acreditar o uso de \textit{CFD} continuará a crescer, limitado apenas pela criatividade
das suas aplicações. E não há razão para acreditar que essa última encontrará barreiras
facilmente.

%Nesse capítulo foram apresentados alguns trabalhos sobre o tema do acoplamento neutrônico/termo-hidráulico 
%com diversas aplicações. Alguns desses trabalhos, são referências no trabalho que está sendo desenvolvido 
%nesta tese e cuja proposta será apresentada com detalhes no próximo capítulo.



%Teste de listagem de programa.
%\lstinputlisting[language=c++, firstline=127, lastline=143]{/home/vitors/workspace/thesisChtMultiRegionFoam/chtMultiRegionSimpleFoam/solid/createSolidFields.H}

%Teste copiando o texto em si.
%\lstset{customc++}
%\begin{lstlisting}
%  if(Pstream::parRun())
%  {
%    fileName cellCorr(runTime.rootPath()+"/"+runTime.caseName()+
%    "/constant/"+solidRegions[i].name()+"/polyMesh");
%    
%    // Read cellProcAddressing for each processor
%    solidList[i][Pstream::myProcNo()] = labelIOList(IOobject
%    (
%        "cellProcAddressing", // Filename
%        cellCorr,
%        solidRegions[i],	// Registry
%        IOobject::MUST_READ, 	// Read option
%        IOobject::NO_WRITE 	// Write Option
%    ));
%    
%    // Create a mapping from each processor to the solidListsRegions
%  }
%\end{lstlisting}


%Tesse de referências usando o \textsf{abntex2} e o 
%pacote \textsf{abntex2cite}itando
%\cite{Janosy2011}. Fala de simulação de forma geral, talvez a ser citado na metodologia.



%\cite{Baglietto2011}.

%Citar \cite{Gleicher2014} que faz acoplamento Bison-Rattlesnake.
%\cite{Beaudoin2008}. CITADO
% Sobre TRIGA tem o trabalho do 
%\cite{Khan2011}. O turco fez benchmarks sobre o TRIGA dele em 
%\cite{Turkmen2013}. Artigo mais matemático, acoplamento totalmente 
%implícito pelo \cite{Pope2008}. O fodão da área é o \cite{Ivanov2007}.
%Os caras da Westinghouse mexem na grade do André \cite{Yan2011} 
%enquanto os iranianos trabalham com o cálculo de parâmetros 
%cinéticos em \cite{Jahanbin2012}. É importante lembrar que um dos 
%autores de \cite{Barber98} é autor de outro dos trabalhos citados. 
%O nome dele é \textit{Downar}. Artigo sobre modelagem \textit{CFD} do TRIGA IPR-R1 
%em \cite{Martinez2012}. Outro paper que descreve o uso de códigos diversos 
%para modelagem e estudo do núcleo de um PWR é: \cite{Huda2011}. 

%\cite{Ragusa2009} é um dos trabalhos com maior detalhamento do 
%acoplamento, focando em controle do \textit{time-step}. Os chineses 
%fizeram um acoplamento e falaram da geometria e usaram fluxograma 
%em \cite{Yang2011}. Um acoplamento entre o MCNPX e o COBRA foi 
%feito em \cite{Vazquez2012}. Um dos poucos trabalhos em que 
%é feita a análise da propagação de erros em processos de software 
%é: \cite{Sarshar2011}. Um trabalho focado na matemática das 
%equações diferenciais da termo-hidráulica é o \cite{Mousseau2007}.
%Vários autores citados em outros trabalhos nesta tese estão juntos 
%em \cite{Barber99} descrevendo os primeiros acoplamentos entre 
%PARCS, RELAP5 e TRAC-M. Uma parte do relatório da comissão regulatória 
%norte-america \cite{NUREG2010} aponta a importância de investir 
%tempo em termo-hidráulica e neutrônica, o papel do regulador nisso 
%e dá justificativas para o tema dessa tese.

%\cite{Aumiller2001} propõe 
%um acoplamento explícito RELAP/\textit{CFD} e conclui que o \textit{CFD} é capaz de 
%calcular a multifísica. A história do código WIMS e a descrição 
%da sua versão 9 estão em \cite{Newton2002}. Nesse trabalho seus 
%resultados são comparados com um código de Monte Carlo. Um paper 
%muito difícil (conteúdo matemático pesado) mas também muito importante 
%é o do \cite{Demaziere2011}. Nele é proposto um código neutrônico 
%multi-propósito, para uso em pesquisa e educação. O trabalho 
%de \cite{Pourgol-Mohamad2011} é focado no tratamento estruturado 
%do modelo de incerteza em códigos neutrônicos e termo-hidráulicos. 
%Citando os livros, o primeiro é o livro base para uso de \textit{CFD}: 
%\cite{Versteeg2007}, no qual são dadas as bases da técnica.
%Outro livro sobre \textit{CFD}, com uma abordagem básica da matemática 
%é o \cite{Anderson95}. Um referência importante no trabalho 
%com o MPI é o livro do \cite{Quinn2004}. E a base para o trabalho 
%de neutrônica está em \cite{Hebert2009}.

%Na elaboração da tese são usados diversos softwares. As informações são 
%geralmente obtidas dos manuais. O código de Monte Carlo com potencial 
%de ser usado é o \cite{Serpent2013}. O código mais importante de 
%acordo com o projeto de tese é o PARCS. Seu manual está divido em 
%manual do usuário \cite{PARCS2006}, manual do programador \cite{PARCS2004} 
%e o manual de teoria \cite{PARCS2004b}. O OpenFOAM é o outro principal 
%software utilizado nesta tese. São também três os manuais usados, sendo 
%o primeiro \cite{OpenFOAM2013} o do usuário, o segundo o guia do 
%programador \cite{OpenFOAM2013b} e a documentação do código 
%em formato eletrônico \cite{OpenFOAM2015}.
