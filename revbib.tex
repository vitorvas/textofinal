% ----------------------------------------------------------
% Revisão Bibliográfica
% ----------------------------------------------------------
\chapter*[Revisão Bibliográfica]{Revisão Bibliográfica}
\label{chap:rev}
\addcontentsline{toc}{chapter}{Revisão Bibliográfica}


%O acoplamento CFD-neutrônica. Motivação. Vantagens e desvantagens. 
%Nesse capítulo vão todas as referências importantes com explicações 
%de cada uma. 
%\section*{OpenFOAM}
%Por que usá-lo? Vantagens e desvantagens.

Antes de passar à descrição do problema de acoplamento, é importante contextualizar sua 
aplicação e as razões que tornam o problema de acoplamento fundamental na simulação completa 
de um reator nuclear .

O crescimento na capacidade de cálculo dos computadores pessoais permitiu não só o uso 
de técnicas mais elaboradas de cálculos termo-hidráulicos bem como seu uso em conjunto 
com códigos de cálculo neutrônico. Um dos primeiros trabalhos diretamente focados no 
acoplamento termo-hidráulico foi apresentado por \cite{Barber98} em 1998. Esse trabalho 
é essencialmente um relatório técnico, no qual o foco está na construção de uma interface 
genérica de acoplamento do código PARCS com códigos de termo-hidráulica. Apesar de não 
gerar resultados de simulações, esse documento evidencia a importância e o interesse 
no procedimento de acoplamento, além de fornecer preciosas informações acerca da 
implementação de estruturas de dados, manipulação de entrada e saída, troca de mensagens, 
uso de biblioteca de troca de mensagens PVM \cite{Geist94}, dentre outros pontos cruciais
da implementação de software. Cabe ressaltar que esta interface, 
chamada \textit{General Interface} é ainda usada pelo código PARCS nas suas versões mais 
recentes. 

Como mencionado no parágrafo anterior, o crescimento na capacidade de processamento
computacional 
fez com que os pesados cálculos de CFD passassem a ser atraentes 
para a Engenharia Nuclear. Mais do que isso, o uso desses códigos passou a ser razão 
de preocupação no sentido de garantir a validade dos seus resultados. O relatório 
da Comissão Regulatória Nuclear (NRC) dos Estados Unidos de 2010 \cite[p.69]{NUREG2010}, 
ao sugerir as melhores práticas e métodos para a atividade de regulação, é bastante claro 
ao apontar a necessidade de tirar proveito da capacidade oferecida por códigos de CFD, 
fornecendo inclusive sugestões de parceria com a indústria e universidades objetivando 
desenvolver simulações multidimensionais devidademente acompanhadas de validação e
verificação. O conteúdo desse relatório é, por si só, prova de que a utilização de 
códigos CFD é, em definitivo, parte do processo de 
pesquisa e desenvolvimento de reatores nucleares. 

Um exemplo desse uso é a utilização de CFD - nesse caso um código proprietário: 
\textit{ANSYS-CFX} - para modelar o fenômeno de ebulição subresfriada para o 
desenvolvimento de combustíveis nucleares \cite{Krepper2007}. Devido à capacidade 
dos códigos CFD de simular detalhes do escoamento de acordo com a granularidade com que se
divide o domínio de simulação, nesse trabalho esta técnica é utilizada para avaliar 
o fenômeno de fluxo de calor crítico em um elemento combustível. As informações 
fornecidas pelo código CFD em relação a fenômenos como rotação, \textit{cross flow} entre 
regiões adjacentes e concentração de bolhas (no caso em que se simule duas fases)
permitem identificar \textit{hot spots}, ou seja, regiões de especial interesse.
Essas simulações permitem avaliar o comportamento
de diferentes projetos de grades espaçadoras de elementos combustíveis quanto ao aspecto 
de segurança.

Um trabalho em conjunto entre pesquisadores do ISYRIM, UFMG e CDTN foi conduzido
com o objetivo de investigar o comportamento do reator TRIGA IPR-R1 por um código do tipo CFD \cite{Martinez2012}. 
Nesse caso não houve tentativa de acoplamento, mas apenas a simulação simplificada 
do reator utilizando-se o \textit{ANSYS-CFX}. Um fluxo de calor 
foi fornecido para as paredes do combustível obedencendo à uma distribuição axial 
característica do combustível. Na análise quantitativa dos resultados, os autores 
informam que os resultados da simulação numérica apresentam boa concordância com 
dados experimentais coletados durante a operação do reator.

%O acoplamento propriamente dito é feito muitas vezes com um código neutrônico não-determinístico. 
%Esses códigos são baseados no método de Monte Carlo(\cite{mc}) e simulam o comportamento 
%do núcleo do reator através da simulação das histórias dos nêutrons. Isso consiste, basicamente, 
%em cálculos de probabilidades de absorção, choque, escape, etc. de um conjunto de nêutrons. 
%Um exemplo de trabalho de simulação do reator TRIGA IPR-R1 pelo método de Monte Carlo está em \cite{Silva2011}.
%Nesse caso, apenas a neutrônica é simulada.

Entretanto, para obter resultados o mais realistas possível, é necessário simular 
tanto os fenômenos termo-hidráulicos quanto neutrônicos. Isso 
se deve ao fato de que os principais fatores que influeciam na distribuição do fluxo de 
nêutrons no núcleo de um reator nuclear são as propriedades do moderador e refrigerante. No caso do 
reator TRIGA, a água exerce ambos os papéis, sendo também usada grafita para 
a moderação. Pequenas variações na temperatura, densidade e composição 
da água podem mudar consideravelmente o fator de multiplicação neutrônico ($k_{eff}$). Daí a importância e 
a necessidade de acoplar os cálculos termo-hidráulicos - as variações da água - com as variações 
no fluxo neutrônico.

O acoplamento das simulações de diferentes fenômenos físicos tem sido também chamado de \textbf{multifísica}. 
Uma definição prática para multifísica ou física-acoplada é dada por \cite{Lethbridge2005}, definindo que 
a multifísica é, em essência, a análise de fenômenos físicos distintos de forma combinada. Nesse texto, 
por conveniência, a palavra acoplamento será usada tanto em referência ao processo de análise dos fenômenos físicos conjuntamente 
quanto em referência à implementação do software para essa tarefa. Nos casos em que possa haver ambiguidade, 
a definição será explicada afim de não deixar margens a dúvidas.

Vale ressaltar, já que nos referimos à implementação do acoplamento em software, que uma aplicação multifísica 
pode levar entre 4 e 6 anos para ser efetivamente útil, podendo chegar a uma vida útil de várias décadas 
\cite{Graham2004}. Isto posto, é possível concluir que o esforço em construir o acoplamento é compensado 
pela possibilidade de uso do código durante vários anos.

% Citar os trabalhos do IVANOV aqui...
Uma vez definido o significado do acoplamento no contexto desse trabalho, serão apresentados os mais importantes 
trabalhos sobre o tema. Talvez o mais importante trabalho relacionado ao acoplamento 
neutrônico/termo-hidráulico seja o de \cite{Ivanov2007}. Nele são análisados e classificados diferentes tipos de 
acoplamento, seus componentes e suas aplicações. Os diversos parâmetros do acoplamento são esquematicamente dividos em: 
\begin{enumerate}
\item Forma de acoplamento: externo quando o código neutrônico é combinado com parte do código termo-hidráulico, 
geralmente na forma de condições de contorno, e então acoplado ao sistema termo-hidráulico completo. O acoplamento 
é definido como interno quando a neutrônica é integrada ao modelo de transferência de calor do sistema termo-hidráulico. 
Em outras palavras, o acoplamento interno é uma implementação multifísica.
\item Abordagens de acoplamento: integração em série ou processamento em paralelo. Na integração em série o algoritmo 
para cálculo neutrônico é integrado como uma rotina ao sistema termo-hidráulico e então executado sequencialmente 
após os cálculos termo-hidráulicos. Na abordagem de processamento em paralelo, geralmente a troca de dados é 
intermediada por um sistema de troca de mensagens como PVM \cite{Geist94} ou MPI \cite{Quinn2004}. Nesse caso, os 
módulos de neutrônica e termo-hidráulica têm capacidade separada de execução e resposta, o que permite uma execução 
mais eficiente do ponto de vista computacional.
\item Sobreposição espacial de malhas: pode ser fixo, quando um canal termo-hidráulico representa um canal neutrônico, 
ou flexível, quando são usados ou especificados esquemas de mapeamento. Um esquema avançado de mapeamento 
e interpolação de malhas \cite{Beaudoin2008} é implementado em alguns dos modernos códigos cd CFD. As diversas implicações 
do mapeamento de malhas para a simulação serão apresentadas oportunamente.
\item Algoritmos de controle de \textit{time-step}: os transientes neutrônicos são geralmente muito mais rápidos que 
os transientes termo-hidráulicos. A utilização de um único \textit{time-step} longo pode levar à não detecção de transientes 
rápidos e no caso oposto ao desperdício de recursos computacionais ao se simular eventos indistinguíveis repetidamente.
\item Acoplamento numérico: o autor se refere nesta classificação ao tempo de troca de informações entre o modelo 
neutrônico e termo-hidráulico. Pode ser explícito, semi-explícito e implícito, dependendo da forma como cada esquema 
realiza o cálculo das variáveis do \textit{time-step} atual baseado em variáveis do \textit{time-step} atual ou 
anterior.
\item Esquemas de convergência do acoplamento: definidos de acordo com a forma com que a simulação é considerada 
finalizada. Se há apenas uma estimativa para a convergência de ambos os códigos, o esquema é considerado fracamento 
acoplado, por exemplo.
\end{enumerate}

O autor ainda comenta a importância da geração de seções de choque adequadas aos transientes esperados na 
simulação acoplada e sua interdependência. A geração de seções de choque não é escopo desta tese, mas uma 
breve explicação de como seções de choque para poucos grupos podem ser geradas pode ser encontrada 
em \cite{Friedman2013}.

Um recente trabalho de simulação envolvendo neutrônica e termo-hidráulica com uso de CFD propõe 
a simulação de reatores avançados refrigerados a gás \cite{Hossain2011}. O modelo utilizado 
para a simulação neutrônica foi o de cinética pontual devido à sua simplicidade. No modelo 
de cinética pontual assume-se que a forma do perfil do fluxo de nêutrons durante um transiente 
não varia, mas apenas os valores do fluxo variam com o tempo. A variação no fluxo é dada 
pelo balanço entre nêutrons produzidos e perdidos, considerando 6 classes de precursores 
para nêutrons atrasados. O desenvolvimento matemático leva a um sistema 7 equações diferenciais 
ordinárias, conhecidas como equações de cinética pontual. A implementação da solução desse sistema 
é, nesse caso, feito dentro do código termo-hidráulico utilizado (\textit{TH3D}).

Recentemente \cite{Yan2011} simulou o comportamento do escoamento em um feixe de elementos 
combustíveis e espaçadores utilizando-se de CFD acoplado à neutrônica. A neutrônica 
simulada é baseada no \textit{method of characteristic}, que evita a necessidade de geração 
de constantes para poucos grupos \textit{a priori}. A abordagem usada no 
acoplamento é do tipo externa, com o código neutrônico DeCART e o código termo-hidráulico 
STAR-CCM+ executando sequencialmente e escrevendo e lendo arquivos em disco em um 
diretório comum. São dois critérios para finalizar a simulãção: o código DeCART se baseia 
na convergência da fonte de fissão, enquanto o código STAR-CCM+ no resíduo da energia. No que toca à 
discretização espacial (sobreposição espacial de malhas), é usado um mapeamento entres as malhas 
utilizadas nos dois códigos. Apesar de diferenças nas malhas, as fronteiras materiais, ou seja, geometria, 
fronteira e posição de diferentes materias são equivalentes em ambos os modelos.

As conclusões apresentadas reiteram a importância da técnica cd CFD na indústria nuclear. Em particular, 
a capacidade de verificar os efeitos da grade misturadora na temperatura e densidade do refrigerante 
levando-se em consideração os efeitos neutrônicos. Nas palavras do autor, esse acoplamento 
``permite um melhor entendimento da margem de DNB (Departure Nuclear Boiling) e a formação de particulado 
na vareta de combustível''. O autor finaliza apontando o interesse em incrementar a simulação 
agregando outros aspectos físicos, como um modelo de corrosão, um modelo de geração do particulado, 
um modelo de interação revestimento/pastilha e outros. Percebe-se aqui que as diversas aplicações 
de multifísica já são esperadas e inevitáveis, e num futuro mais do que próximo.

Talvez o mais completo trabalho relativo ao acoplamento neutrônico-termo-hidráulico com uso do 
CFD \textit{OpenFOAM} seja \cite{Jareteg2012}. Nesta tese, o acoplamento é feito usando o mesmo 
software para a simulação termo-hidráulica e neutrônica. A metodolgia utilizada pode ser brevemente 
resumida em alguns passos principais: 1) criação de uma malha adequada, 2) discretização das 
equações descrevendo o problema, 3) definição das condições de contorno, 4) geração dos 
parâmetros neutrônicos (por exemplo, seções de choque macroscópicas) e 5) solução do 
problema neutrônico e termo-hidráulico de forma acoplada. O autor usa um dos modelos de solução presentes no 
software de CFD \textit{OpenFOAM} \cite{OpenFOAM2013} e 
altera esse modelo que simula um escoamento turbulento de um fluido compressível com transferência de por radiação 
térmica (\texttt{buoyantSimpleRadiationSolver}) adicionando uma implementação da equação de difusão multi-grupos 
ao modelo. Isto é feito utilizando a capacidade geral de discretização e solução de equações bem como os algorimos 
para solução numérica já presentes no \textit{OpenFOAM}. Dentre suas conclusões, estão que o acoplamento iterativo 
utilizado é funcional e estável, a solução para a neutrônica necessitou de menos iterações para convergir 
do que que a solução pressão-velocidade e que a termo-hidráulica precisa de malhas mais refinadas do que a solução 
neutrônica. A qualidade desse trabalho e muitas das soluções adotadas servem como referências às implementações 
e simulações a serem feitas nesta tese.


%o acoplamento é interno ao mesmo software, 
%tendo sido a neutrônica simulada pela solução da equação de Difusão. Nesse caso o acoplamento 
%tem por objetivo simular um reator de potência do tipo PWR.

Uma das razões do uso do acoplamento, cujo objetivo é obter dados mais precisos e realistas, é na 
análise de segurança de reatores. Em projetos de reatores inovadores já são feitos cálculos de multifísica para 
análise de acidentes. No trabalho de \cite{Lazaro2013}, um reator de IV geração refrigerado a sódio 
é simulado de forma acoplada. Códigos já usados na análise de reatores resfriados a água leve são adaptados 
para o uso em reatores de nêutrons rápidos resfriados a sódio. Um ponto-chave, segundo o autor, é ser capaz 
de simular os fenômenos presentes na operação de reatores em três dimensões. Isso se justifica, ainda nas 
palavras do autor, devido a possíveis componentes assimétricos em transientes hipotéticos e que 
a modelagem unidimensional com cinética pontual não é capaz de reproduzir. Assim como em vários trabalhos 
sobre o tema de acoplamento, o código usado para a geração de seções de choque homogeneizadas 
foi o \textit{Serpent} \cite{Serpent2013}. Nesse trabalho, o código \textit{Serpent} foi ainda usado 
para cálculo neutrônico do núcleo e validação do código PARCS \cite{PARCS2006}. Sua conclusão nesse trabalho 
é de que as adaptações feitas nos códigos levaram a resultados consistentes, mas que ainda há trabalho 
a ser feito para a simulação tridimensional completa desse tipo de reator.

Nesse capítulo foram apresentados alguns trabalhos sobre o tema do acoplamento neutrônico/termo-hidráulico 
com diversas aplicações. Alguns desses trabalhos, são referências no trabalho que está sendo desenvolvido 
nesta tese e cuja proposta será apresentada com detalhes no próximo capítulo.

Teste de algumas referências como \cite{Fiorina2015}.
%Tesse de referências usando o \textsf{abntex2} e o 
%pacote \textsf{abntex2cite}itando\cite{Larsson2012}. 
%O próximo é o Janosy \cite{Janosy2011}. Depois vem o artigo 
%do finlandês \cite{Leppanen2012} e mais um capítulo de 
%livro \cite{Faghihi2011}. E o autor do Openfoam \cite{Jasak2007}. 
%Um artigo sobre as novas perspectivas do desenvolvimento de 
%reatores em \cite{Baglietto2011}. Condições de contorno pra acoplamento 
%no OpenFOAM em \cite{Beaudoin2008}. Sobre TRIGA tem o trabalho do 
%\cite{Khan2011}. O turco fez benchmarks sobre o TRIGA dele em 
%\cite{Turkmen2013}. Artigo mais matemático, acoplamento totalmente 
%implícito pelo \cite{Pope2008}. O fodão da área é o \cite{Ivanov2007}.
%Os caras da Westinghouse mexem na grade do André \cite{Yan2011} 
%enquanto os iranianos trabalham com o cálculo de parâmetros 
%cinéticos em \cite{Jahanbin2012}. É importante lembrar que um dos 
%autores de \cite{Barber98} é autor de outro dos trabalhos citados. 
%O nome dele é \textit{Downar}. Artigo sobre modelagem CFD do TRIGA IPR-R1 
%em \cite{Martinez2012}. Outro paper que descreve o uso de códigos diversos 
%para modelagem e estudo do núcleo de um PWR é: \cite{Huda2011}. 
%\cite{Ragusa2009} é um dos trabalhos com maior detalhamento do 
%acoplamento, focando em controle do \textit{time-step}. Os chineses 
%fizeram um acoplamento e falaram da geometria e usaram fluxograma 
%em \cite{Yang2011}. Um acoplamento entre o MCNPX e o COBRA foi 
%feito em \cite{Vazquez2012}. Um dos poucos trabalhos em que 
%é feita a análise da propagação de erros em processos de software 
%é: \cite{Sarshar2011}. Um trabalho focado na matemática das 
%equações diferenciais da termo-hidráulica é o \cite{Mousseau2007}.
%Vários autores citados em outros trabalhos nesta tese estão juntos 
%em \cite{Barber99} descrevendo os primeiros acoplamentos entre 
%PARCS, RELAP5 e TRAC-M. Uma parte do relatório da comissão regulatória 
%norte-america \cite{NUREG2010} aponta a importância de investir 
%tempo em termo-hidráulica e neutrônica, o papel do regulador nisso 
%e dá justificativas para o tema dessa tese. \cite{Aumiller2001} propõe 
%um acoplamento explícito RELAP/CFD e conclui que o CFD é capaz de 
%calcular a multifísica. A história do código WIMS e a descrição 
%da sua versão 9 estão em \cite{Newton2002}. Nesse trabalho seus 
%resultados são comparados com um código de Monte Carlo. Um paper 
%muito difícil (conteúdo matemático pesado) mas também muito importante 
%é o do \cite{Demaziere2011}. Nele é proposto um código neutrônico 
%multi-propósito, para uso em pesquisa e educação. O trabalho 
%de \cite{Pourgol-Mohamad2011} é focado no tratamento estruturado 
%do modelo de incerteza em códigos neutrônicos e termo-hidráulicos. 
%Citando os livros, o primeiro é o livro base para uso de CFD: 
%\cite{Versteeg2007}, no qual são dadas as bases da técnica.
%Outro livro sobre CFD, com uma abordagem básica da matemática 
%é o \cite{Anderson95}. Um referência importante no trabalho 
%com o MPI é o livro do \cite{Quinn2004}. E a base para o trabalho 
%de neutrônica está em \cite{Hebert2009}.

%Na elaboração da tese são usados diversos softwares. As informações são 
%geralmente obtidas dos manuais. O código de Monte Carlo com potencial 
%de ser usado é o \cite{Serpent2013}. O código mais importante de 
%acordo com o projeto de tese é o PARCS. Seu manual está divido em 
%manual do usuário \cite{PARCS2006}, manual do programador \cite{PARCS2004} 
%e o manual de teoria \cite{PARCS2004b}. O OpenFOAM é o outro principal 
%software utilizado nesta tese. São também três os manuais usados, sendo 
%o primeiro \cite{OpenFOAM2013} o do usuário, o segundo o guia do 
%programador \cite{OpenFOAM2013b} e a documentação do código 
%em formato eletrônico \cite{OpenFOAM2013c}.
