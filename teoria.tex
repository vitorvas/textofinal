% ----------------------------------------------------------
% Teoria
% ----------------------------------------------------------
\chapter{Teoria}
\label{chap:teoria}

O trabalho desenvolvido nesta tese tem como base métodos de
solução numérica de equações diferenciais parciais. Neste capítulo, são
apresentadas as bases teóricas da modelagem matemática
dos problemas e dos métodos de solução utilizados.


%\begin{figure}[htb]
%  \caption{Teoria: o sistema acoplado.}
%  \centering\includegraphics[scale=0.7]{figuras/teoria1.png}
%  \label{metodoetapas}
%  \legend{Fonte: autor}
%\end{figure}

% **********************************************
\section{Termo-hidráulica}
\label{sec:th}

A Termo-hidráulica é a área de estudo dos de transferência
de calor e massa, processos fluido-mecânicos com transporte de energia e
massa em sistemas nucleares. Os principais fenômenos estudados incluem condução,
convecção, transferência de calor por radiação, mudanças de fase e escoamentos
monofásicos e multifásicos.

A análise termo-hidráulica de sistemas de conversão de energia envolve a solução
das equações de transporte de massa, momento e energia \cite{Todreas2012}. Nesta seção serão apresentadas
as equações utilizadas na modelagem de um sistema termo-hidráulico monofásico e
estacionário.

% **********************************************
\subsection{Equações governantes}
\label{subsec:eq}


% **********************************************
%\subsubsection{Fluidos}
%\label{ssubsec:fluid}

% **********************************************
\subsection{Turbulência}
\label{subsec:}

% **********************************************
\subsection{Modelo termo-hidráulico discretizado}
\label{subsec:modeloth}


% **********************************************
%\subsubsection{Sólidos}
%\label{ssubsec:solid}

% **********************************************
\section{Neutrônica}
\label{sec:neutronica}

% **********************************************
\subsection{Dados nucleares}
\label{subsec:dn}

% **********************************************
\subsection{Métodos estocásticos}
\label{subsec:mc}

% **********************************************
\subsection{Métodos determinísticos}
\label{subsec:det}

% **********************************************
\subsubsection{Equação de Transporte}
\label{ssubsec:transp}

% **********************************************
\subsubsection{Aproximação por Difusão}
\label{ssubsec:difusao}

% **********************************************
\subsection{Modelo neutrônico discretizado}
\label{subsec:modelon}
