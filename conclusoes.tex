% ----------------------------------------------------------
% Conclusões
% ----------------------------------------------------------
\chapter{Conclusões}
\label{chap:conclusoes}

Nesta tese foi desenvolvido um sistema livre, aberto e gratuito para
cálculos neutrônicos e termo-hidráulicos de forma acoplada utilizando
malhas idênticas. Além do sistema propriamente dito, foi também desenvolvida
uma metodologia para a construção deste sistema acoplado baseada em sistemas
já existentes livres e utilizando memória compartilhada como forma de
intercâmbio de dados.

Este sistema é inovador ao utilizar o \textit{framework OpenFOAM}, baseado
em volumes finitos, como ferramenta de cálculos termo-hidráulicos e o
código de cálculos de física de reatores aberto \textit{milonga} como ferramenta
para cálculos neutrônicos. A utilização de memória compartilhada como
espaço de intercâmbio de dados, apesar de conhecida \cite{Maciel2011, Theler2013},
é praticamente ignorada pela comunidade de Engenharia Nuclear Computacional na
implementação de sistemas acoplados. Tal situação é de se estranhar, já que o
uso de memória compartilhada permite a comunicação entre os sistemas acoplados
de forma confiável, robusta e tão rápida quanto qualquer acesso a memória
do computador.

O sistema desenvolvido, por acoplar dois sistemas de cálculos que utilizam
a formulação de volumes finitos para solução de sistemas de equações diferenciais,
explora a característica de domínio idêntico para os dois sistemas. Em outras palavras,
os domínios de solução, ou malhas, são os mesmos para ambos o que evita a utilização
de funções de mapeamento, ou seja, de todo o \textit{overhead} mapeamento geométrico.

Há, naturalmente, vantagens e desvantagens no uso de malha idêntica. Uma desvantagem
é o custo computacional para malhas refinadas. Além disso, o fluxo neutrônico geralmente
não precisa ser conhecido em detalhes na maioria dos cálculos de núcleo de reatores,
sendo esta uma crítica bem fundada. Entretanto, o objetivo dos cálculos por volumes
finitos é, também geralmente, permitir identificar pequenas ocorrências imperceptíveis
em cálculos por métodos menos granulares. O sistema desenvolvido nesta tese, cuja
prova conceitual foi feita com uma malha relativamente pobre, permite conhecer fenômenos
locais calculados a partir de elementos da multi-física envolvida, e não apenas
de perfis de potência genéricos ou temperaturas médias, por exemplo.

O desenvolvimento desta tese levou também a contribuições fora do seu escopo principal mas,
certamente, profundamente relacionadas à metodologia de desenvolvimento. Foram encontradas
falhas de implementação em ambos os sistemas utilizados para neutrônica e termo-hidráulica,
sendo os erros no \textit{OpenFOAM} reportados para a comunidade (e eventualmente
corrigidos em outras versões) e os erros no \textit{milonga} não só reportados ao seu autor como,
em um caso crítico, feita a correção diretamente e enviado o \textit{patch}. Estas contribuições
incidentais, se dão devido à própria natureza do desenvolvimento baseado
em \textit{software} livre, que prevê e se baseia na contribuição e troca entre a comunidade
de usuários.

É dessa corrente de desenvolvimento que surge outra, e talvez a principal, característica inovadora
do sistema desenvolvido nesta tese: ele é construído exclusivamente a partir
de sistemas abertos e livres\footnote{A metodologia utilizada
  na geração de seções de choque utiliza um \textit{software} restrito. Entretanto, a geração
  de seções de choque, apesar de fundamental para os cálculos neutrônicos, não é, propriamente dita,
  parte do sistema acoplado, já que qualquer ferramenta pode ser utilizada para este propósito. Neste tema, começam também a surgir opções em \textit{software} livre
  para a geração de seções de choque em multi-grupos a partir de bibliotecas contínuas \cite{pyne2014}.}.
Esta característica e suas implicações merecem uma seção exclusiva.


\section{Discussão sobre \textit{software} livre}

A princípio, o fato de um \textit{software} ser distribuído de forma livre ou restrita pode soar indiferente.
Entretanto, numa visão mais cuidadosa, é fácil
perceber que a liberdade na utilização, modificação e distribuição de um \textit{software} trás imediatamente
diversos benefícios ao seu utilizador e, consequentemente, à comunidade de usuários. O primeiro
benefício é financeiro. Não é necessário dispender dinheiro em licenças de uso, autorizações de uso ou
qualquer outro aspecto do uso de \textit{software}. A independência financeira é, em último caso,
ainda mais importante para países em desenvolvimento que, nem sempre, conseguem garantir um fluxo constante
no aporte de financiamento a instituições de pesquisa e acadêmicas. O Brasil entra, obviamente, nesta lista.

A liberdade de utilização está também no acesso ao código-fonte. A capacidade de estudar, entender, modificar e experimentar
com o \textit{software} está totalmente ligada ao acesso a como se dá seu funcionamento. É novamente fácil perceber que,
do ponto de vista tecnológico, o acesso ao código-fonte funciona como uma transferência de tecnologia. Além disso, no
aspecto educacional, é uma forma de colocar as novas e futuras gerações de profissionais em contato com a tecnologia na
prática, para além das abordagens didáticas (e por vezes limitadas) teóricas. 


%\begin{figure}[htb]
%  \caption{Conclusões: o sistema acoplado.}
%  \centering\includegraphics[scale=0.7]{figuras/conclusoes1.png}
%  \label{metodoetapas}
%  \legend{Fonte: autor}
%\end{figure}

\section{Trabalhos Futuros}
