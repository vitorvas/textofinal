% ----------------------------------------------------------
% Metodologia
% ----------------------------------------------------------
\chapter{Metodologia}
\label{chap:metodologia}

O trabalho de tese em questão trata-se, essencialmente, do desenvolvimento e testes
de um sistema de \textit{software}. Desse modo, parece desejável, do ponto de vista de
clareza, descrever a metodologia utilizada no trabalho de forma a acompanhar o ciclo
de desenvolvimento dos vários elementos de software utilizados. Para isso, a metodologia
será divida em seções, da seguinte forma:
\begin{enumerate}
\item Termo-hidráulica;
\item Neutrônica;
\item Acoplamento;
  \item Validação.
  \end{enumerate}

Como todos os elementos do desenvolvimento estão relacionados entre si, sempre que
julgado oportuno para a clareza do entendimento, os conceitos desenvolvidos em cada
seção poderão ser apresentados conjuntamente. De fato, uma vez que o problema multifísica
(ou acoplamento) nada mais é do que a solução conjunta de dois problemas que são,
usualmente, resolvidos separadamente, é natural como definir, modelar e resolver os
dois problemas separadamente.

De acordo com as definições de problemas acoplados apresentadas
no capítulo \ref{chap:rev}, ao resolver os problemas neutrônico e termo-hidráulico
separadamente, o acoplamento ainda pode ser considerado implícito de acordo com, já que a troca de informações entre os problemas ocorre ao nível de termos completos,
e não em condições de contorno. É importante dizer que essa nomeclatura para as formas
de acoplamento não é unânime na literatura \cite{Ivanov2007}.

Deve estar claro que a metodologia a ser descrita, trata do ciclo de desenvolvimento
do \textit{software} acoplado. Apesar de ser possível considerar as simulações e
os processos de pré-processamento e pós-processamento, tanto termo-hidráulicos quanto
neutrônicos, optou-se por tratar destas etapas exclusivamente na seção de validação e,
quando pertinente, nos resultados obtidos.

\section{Termo-hidráulica}
\label{sec:th}

Para a realização dos cálculos termo-hidráulicos, é usado o \textit{software OpenFOAM},
um sistema de \textit{CFD}.
