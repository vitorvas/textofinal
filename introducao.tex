% ----------------------------------------------------------
% Introdução
% ----------------------------------------------------------
\chapter{Introdução}
\label{chap:introducao}

\section{Contextualização}

\emph{``Over a 20-year period, the (nuclear) industry will move from 
the application of conventional methods that rely on 
experimental correlations to using CFD (...)''.} \cite[p.~655]{Baglietto2011}.

A frase acima é um exemplo de ganho de importância da técnica de dinâmica dos fluidos 
computacional (CFD) na indústria nuclear nos últimos anos e da perspectiva da sua aplicação 
nos próximos. Uma das aplicações em que o uso de CFD tem se estendido é no cálculo termo-hidráulico
de combustíveis e de elementos combustíveis nucleares. Em boa parte dos trabalhos seminais (CITAR), dentre
as condições iniciais e de contorno do problema termo-hidráulico, se define um valor de potência inicial,
ou mesmo variável, à qual está submetido o sistema. Apesar de extremamente útil, essa forma de prescrever a
potência é uma forma de simplificação. Isso é devido às particularidades do forma de geração de energia nuclear:
a cinética dos nêutrons, ou neutrônica.

A disponibilidade do código-fonte é condição básica para o desenvolvimento do sistema acoplado. Grosso modo, o acoplamento
dos dois códigos nada mais é do que alterações e adições em ambos os códigos-fonte de modo
que estes sejam capazes de se comunicar e trocar dados para realizar seus cálculos.
Este processo, desde sua metodologia até fundamentos de sua execução, será
oportunamente - e detalhadamente - apresentado, já que trata-se da principal
contribuição desta tese.

O acoplamento entre neutrônica e termo-hidráulica já ocorre de diferentes formas. Na simulação 
termo-hidráulica são usados códigos de sistemas e códigos de sub-canal. Os códigos de sistemas 
funcionam modelando os sistemas do reator unidimensionalmente e aplicando as equações básicas 
para continuidade, momento e energia. O resultado obtido simula o
comportamento médio dos componentes do reator.
Estes códigos são normalmente usados em análises de transientes e segurança de reatores. 
Já os códigos de sub-canal são mais detalhados e, além
de modelar múltiplos componentes do sistema 
do reator, são capazes de simular em geometrias tridimensionais \cite{Faghihi2011}. Os códigos 
CFD substituem o domínio contínuo por um domínio discreto e finito. Neste contexto, as equações 
que governam o escoamento são integradas sobre todos os elementos que formam o domínio (finito). 
As integrais obtidas são então discretizadas na forma de um sistema de equações algébricas 
e, por fim, este sistema é resolvido por métodos interativos \cite{Versteeg2007}.

No contexto da neutrônica, é possível classificar os códigos em três tipos:
1) Códigos de difusão ou de transporte, 
2) Códigos de ordenadas-discretas e 3) Códigos Monte Carlo. Os dois primeiros são determinísticos 
e o terceiro estocástico. 

Os códigos de difusão resolvem a equação de difusão de nêutrons. A equação de difusão de nêutrons nada mais é do
que uma simplificação na modelagem do comportamento dos nêutrons. Uma delas, por exemplo, é a consideração de um
coeficiente de difusão único representando as direções possíveis dos nêutrons. Em suma, a equação de difusão
em estado estacionário é obtida de uma relação entre a corrente de nêutrons e o gradiente do fluxo neutrônico,
representando o fato de que os nêutrons têm uma tendência a migrar de regiões onde são mais numerosos para
regiões onde são menos numerosos \cite{Hebert2009}. Um dos códigos de difusão 
mais usados é o PARCS (\textit{Purdue Advanced Reactor Core Simulator}), estando inclusive já acoplado 
com códigos de termo-hidráulica de sistemas \cite{Xu2006,Barber98}.
O código usado neste trabalho para o acoplamento é o código milonga
(grafa-se sem letra maiúscula). O milonga \cite{Theler2014b}
utiliza o método de volumes finitos na discretização do domínio e disponibiliza a solução
do cálculo neutrônico pela equação de difusão ou pelo método de ordenadas discretas. Suas características e seu
funcionamento serão apresentados oportunamente.

Os códigos de ordenadas discretas resolvem 
a equação de transporte de \textit{Boltzmann} para o comportamento médio das partículas para então calcular o 
fluxo de nêutrons. Nesses códigos, o espaço é divido em muitas pequenas caixas e as partículas 
são movidas entre as caixas. Para geometrias complexas com variações de parâmetros, a preparação de 
seções de choque exige grande esforço.

Os códigos de Monte Carlo funcionam simulando as partículas 
individualmente e gravando aspectos do seu comportamento médio. Devido ao alto custo computacional dos cálculos
pelo método de Monte Carlo, tardaram a ser usados em cálculos acoplados em comparação a métodos determinísticos.
Entretanto, nos últimos anos seu uso em sistemas acoplados aumentou consideravelmente \cite{Herman2015, Richard2015, Bennett2016},
inclusive com o uso de CFD \cite{Leppanen2012}.

Uma vez apresentados os tipos de códigos mais utilizados para simulações termo-hidráulicas
e neutrônicas, o próximo passo é entender o porquê de acoplar estes códigos.

\section{Descrição do trabalho}

Este trabalho de tese consiste no desenvolvimento de um sistema acoplado de cálculos
termo-hidráulicos e neutrônicos. Apesar de desenvolvidos separadamente - inclusive
em linguagens de programação diferentes - ambos fazem uso do método de volumes
finitos \cite{Eymard2003} (FVM em inglês) na solução do seu conjunto particular de equações.
Cabe ressaltar que, neste primeiro momento, apenas cálculos em estado estacionário
são previstos.

Os códigos utilizados são abertos \cite[Capítulo~3]{Stallman2002}, o que significa pleno acesso ao seu código-fonte.
Essa informação poderia passar despercebida, mas há alguns pontos a considerar antes de se
seguir. As licenças que permitem a distribuição de código aberto impõe, via de regra, que
o código desenvolvido a partir de outro licenciado desta forma deva permanecer aberto.
Apesar do grande debate sobre a segurança dos códigos abertos e suas vantagens e desvantagens \cite[Seção~2.6]{Androutsellis2010},
é fato que desde o crescimento no uso do sistema operacional Linux \cite{LinuxBritannica}, sua disseminação é crescente.
Não apenas para tarefas básicas, mas predominantemente em serviços de rede, internet, cálculo numérico dentre
inúmeros outros. Sistemas, códigos ou \textit{frameworks} baseados em códigos abertos são
extensivamente usados em aplicações de missão crítica \cite{Norris2004}.

Dada que a utilização de programas de código aberto ainda é relativamente tímida no domínio nuclear
- e a expressão relativamente talvez seja inapropriada, pois são vários projetos de instituições
renomadas utilizando código aberto \cite{Romano2013, Boyd2014, Huff2016} -
esta tese tem também por objetivo iniciar a discussão sobre as possibilidades de sua
aplicação neste domínio. Sendo os dois códigos usados para o acoplamento no âmbito desta tese disponibilizados
de forma aberta, temos, portanto, um sistema também aberto. Sistema este, facilmente auditável
por qualquer interessado, seja ele desenvolvedor, engenheiro, pesquisador ou regulador.
Se essa forma de desenvolvimento de sistemas de \textit{software} é adequada para as especificidades
e restrições dos sistemas nucleares, não é possível dizer. Não ainda, não antes de que esse debate
seja iniciado.

%------------------------------------------------------------------------------------
%A simulação computacional de fenômenos 
%físicos não é por si só algo novo na indústria nuclear. Entretanto, com o regular 
%crescimento da capacidade computacional, técnicas mais exigentes em termos computacionais 
%passaram a fazer parte do dia-a-dia de projetistas, desenvolvedores e pesquisadores. Neste 
%contexto está a técnica de CFD.
%
%Neste trabalho, propõe-se um passo além do uso desta ferramenta: seu uso para a simulação 
%termo-hidráulica de uma vareta combustível de um reator do tipo PWR acoplada ao seu respectivo cálculo neutrônico. Cabe ressaltar que neste primeiro momento, os cálculos se restringem a
%um estado estacionário, tanto termo-hidráulico quanto neutrônico. Além disso, 
%é de fundamental importância enfatizar que os problemas de termo-hidráulica e neutrônica 
%estão essencialmente ligados. Isso se dá em razão da influência da potência gerada na 
%temperatura e densidade dos materiais constiuintes da vareta combustível, bem como
%do material refrigerante e moderador. Por sua vez, estas variações nas propriedades
%dos materiais influenciam na reprodução de nêutrons, o que faz variar a potência e assim
%por diante.

\section{Motivação}

Pode-se dizer que há duas grandes motivações para a realização deste trabalho. A primeira, de ordem principalmente
técnica, é o uso de um código do tipo CFD acoplado para cálculo multi-física. O CFD permite o cálculo detalhado do
comportamento do escoamento em um reator, subcanal ou ao redor de um elemento. Tal nível de detalhes, permite a
investigação de fenômenos físicos não modeláveis de outras formas - no nível de todo o sistema, por exemplo. Além disso,
o acoplamento entre os fenômenos neutrônico e termo-hidráulico resolve a dependência entre os dois fenômenos de forma
mais fiel ao que realmente ocorre na prática. Obviamente, há um custo para a obtenção deste nível de detalhamento: a
demanda computacional. Felizmente, os atuais processadores e suas características de multi-processamento permitem
a utilização de métodos outrora considerados uma ousadia. Isto posto, é possível afirmar que cálculos multi-física ou
acoplados já são uma realidade e não se pode abrir mão de conhecer os novos aspectos técnicos, numéricos e teóricos
envolvidos nesta nova forma de investigar a física de reatores.

A segunda grande motivação no desenvolvimento deste trabalho, vai além dos aspectos puramente técnicos.
O chamado \textit{software livre} consiste em códigos desenvolvidos dentro de uma filosofia de liberdade e de uso
comunitário. Essa ideia, que remonta ao início dos anos 80, advoga que um programa ou \textit{software} deve fornecer
o código-fonte aos usuários. Os usuários têm direito de modificá-lo, melhorá-lo e até mesmo vendê-lo, desde que o novo
programa seja fornecido também com código-aberto. Essa filosofia de desenvolvimento comunitário de \textit{software}
chocou o mercado de micro-computação quando surgiu, mas cerca de 35 anos depois, não só é uma realidade como
grande parte do \textit{software} em uso para certos tipos de aplicações hoje é baseado em \textit{software livre}
\cite{Androutsellis2010}. Não seria de se esperar outra coisa que esta filosofia de desenvolvimento comunitário de
\textit{software} alcançasse a engenharia nuclear. As discussões sobre quão efetiva será a adoção deste modelo
para uma área de missão crítica, como boa parte das aplicações em física de reatores, é, por si só, assunto para uma
tese exclusiva sobre o tema. Neste trabalho, a motivação está na possibilidade de junção de forças entre grupos com
diferentes \textit{backgrounds}, estruturas de pesquisa e investimentos com um objetivo comum: o desenvolvimento
de sistemas úteis para todo o grupo, revisado por pares - e com isso uma mais ampla rede de detecção de erros
de implementação - e possíveis de serem modificados e alterados de acordo com as necessidades.

Com o acoplamento entre dois códigos abertos que se apresentará nesta tese, apesar
de uma iniciativa pequena, espera-se trazer à luz do dia a discussão sobre o uso de \textit{software livre} na
indústria e pesquisa nucleares entre especialistas no Brasil. Afinal, a filosofia do uso de \textit{software livre}
já começou nos centros de ponta espalhados pelo planeta \cite{Romano2013, Boyd2014, Huff2016}.




%Apesar da existência na literatura de vários trabalhos de acoplamento neutrônico-termo-hidráulico \cite{Faghihi2011}, 
%a utilização de códigos do tipo CFD para a realização dos cálculos 
%termo-hidráulicos ainda é tímida. Isso pode ser explicado pela alta demanda
%computacional, dificultando sua utilização para sistemas da complexidade de um
%reator nuclear. Além disso, até o presente momento, não
%foram encontrados trabalhos
%sobre o uso do \textit{OpenFOAM} - 
%um CFD aberto e gratuito \cite{Jasak2007} - acoplado a códigos de neutrônica.

%Os mais recentes trabalhos utilizando \textit{OpenFOAM} para cálculos acoplados
%não se utilizavam de códigos neutrônicos externos. O capítulo \nameref{chap:rev} apresenta
%uma revisão dos principais trabalhos no tema de acoplamento.


%Além da originalidade do uso do 
%\textit{OpenFOAM} e sua aplicação para o caso simples de uma vareta combustível, uma vez validada pelos dados experimentais de \textit{benchmarks} de reatores
%do tipo PWR\cite{benchmarksPWR}, a metodologia desenvolvida nesta tese poderá ser usada para simulações reatores completos. São eles o reator multipropósito brasileiro, com %previsão de operação para 2020 e reatores de potência tipo PWR em operação no Brasil (Angra I e II). 

%E a aplicação não se restringe à simulação do reator, podendo ser, no futuro, adicionados elementos como queima de combustível, envenenamento por xenônio e outros fatores que permitam não só a simulação de um estado transiente curto, como também da evolução da utilização do reator no tempo.
